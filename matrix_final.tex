\documentclass[12pt]{article}
\usepackage{graphicx}
\usepackage{fullpage}
\usepackage{amsmath}
\usepackage{amsfonts}
\usepackage{amssymb}
\usepackage{amsthm}
\usepackage{algpseudocode}
\usepackage{algorithm}

\newtheorem{proposition}{Proposition}
\theoremstyle{plain}

\begin{document}

\title{ECE 210A: Final Track 2}
\author{Shiyu Ji}
\date{\today}
\maketitle

\newcommand{\m}[1]{\begin{pmatrix}#1\end{pmatrix}}
\newcommand{\rank}[1]{\operatorname{rank}(#1)}

\section{Self Assessment}

\begin{tabular}{c|l}
Problem 1 & 2.\\
Problem 2 & 2.\\
Problem 3 & 2 (including all the 8 questions).\\
Problem 4 & 2.\\
\end{tabular}

\section{Problem 1}
Let $p\in\mathbb{R}^n$ such that $p_i > 0$ for $i=1,\cdots,n$, and $\sum_{i=1}^n p_i = 1$. Show that
$$n^3 + 2n + \frac{1}{n} \leq \sum_{i=1}^n \left(p_i + \frac{1}{p_i} \right)^2.$$

\begin{proof}
Clearly,
$$\sum_{i=1}^n \left(p_i + \frac{1}{p_i} \right)^2 = \sum_{i=1}^n \left( p_i^2 + 2 + \frac{1}{p_i^2} \right) = 2n + \sum_{i=1}^n \left(p_i^2+\frac{1}{p_i^2}\right).$$
To compute the minimum, we use Lagrange multiplier. That is, let
$$\mathcal{L}(p_1,p_2,\cdots,p_n, \lambda) = \sum_{i=1}^n \left(p_i^2+\frac{1}{p_i^2}\right) + \lambda \sum_{i=1}^n p_i^2.$$
Here $\lambda$ can be any real number. 
Then let
$$\frac{\partial \mathcal{L}}{\partial p_i} = 2p_i - \frac{2}{p_i^3} + \lambda =0.$$
Since for any $i$, $p_i>0$, we have for any $i$,
$$2p_i^4 + \lambda p_i^3 -2 = 0.$$
Note that denote $f(p) = 2p^4 + \lambda p^3 - 2$. Then $f(0) = -2$ and $f(1) = \lambda$. And $f'(p) = 8p^3 + 2\lambda p^2 > 0$ if $0<p<1$, i.e., $f$ monotonically increases over the unit interval. Thus if $\lambda > 0$, it is guaranteed that there is a unique maximum point over $0<p_i<1$ for any $i$. Since the above equation holds for any $i$, each $p_i$ takes the same value. Since $\sum_{i=1}^n p_i = 1$, we have 
$$\forall i, p_i = 1/n.$$
Thus
$$\sum_{i=1}^n \left(p_i^2+\frac{1}{p_i^2}\right) \geq \sum_{i=1}^n \left( \frac{1}{n^2} + n^2\right) = n^3 + \frac{1}{n}.$$
Hence we have shown
$$n^3 + 2n + \frac{1}{n} \leq \sum_{i=1}^n \left(p_i + \frac{1}{p_i} \right)^2.$$
The equality holds iff for any $i$, we have $p_i=1/n$.
\end{proof}

\section{Problem 2}
Let $x\in\mathbb{R}^n$. Let $D$ be the $n\times n$ diagonal matrix with diagonal entries $D_{i,i} = (-1)^i$. Let $\lceil t\rceil$ denote the smallest integer bigger than or equal to $t$. Let $e\in \mathbb{R}^n$ such that $e_i=1$ for $i=1,\cdots,n$. Show that
$$|e^Ta|^2 + |e^T Da|^2 \leq 2 \lceil \frac{n}{2}\rceil ||a||_2^2.$$

\begin{proof}
Firstly, we have
$$e^T a = \sum_{i=1}^n a_i,\quad e^T Da = \sum_{i=1}^n (-1)^i a_i.$$
Thus it suffices to show
$$\left( \sum_{i=1}^n a_i \right)^2 + \left( \sum_{i=1}^n (-1)^i a_i \right)^2 \leq 2 \lceil \frac{n}{2}\rceil ||a||_2^2.$$
On the other hand, we have
$$\left( \sum_{i=1}^n a_i \right)^2 + \left( \sum_{i=1}^n (-1)^i a_i \right)^2
=2\sum_{2|(i-j)}a_ia_j.$$
Since $0\leq (a_i-a_j)^2 = a_i^2 + a_j^2 - 2a_ia_j$, we have
$$2\sum_{2|(i-j)}a_ia_j \leq \sum_{2|(i-j)} (a_i^2 + a_j^2).$$
We fix $i$, and see how many pairs $(i,j)$ are there satisfying $2|(i-j)$. If $i$ is odd, then $j$ must be odd, too. Similarly, if $i$ is even, so must be $j$. Hence the number of such pairs must be no more than $\lceil \frac{n}{2}\rceil$. Similar we can apply the same reasoning to $j$:
$$\sum_{2|(i-j)} (a_i^2 + a_j^2)
\leq \sum_{i} \left( \lceil \frac{n}{2} \rceil a_i^2 + \sum_{2|(i-j)} a_j^2 \right)
= \sum_i \lceil \frac{n}{2} \rceil a_i^2 + \sum_i \sum_{2|(i-j)} a_j^2$$
$$\leq \sum_i \lceil \frac{n}{2} \rceil a_i^2 + \sum_j \lceil \frac{n}{2} \rceil a_j^2
= 2\lceil \frac{n}{2} \rceil ||a||_2^2.$$
Thus we have shown the inequality as desired. Based on the proof, it is clear that the equality holds when $n$ is even and each $a_i$ has the identical value.
\end{proof}

\section{Problem 3}
a) Let $T$ be an $n\times n$ Toeplitz matrix. Let $Z_n$ denote the $n\times n$ shift-up matrix. What is the maximum rank of the matrix $TZ_n - Z_nT$?

{\it Solution.} 
Let
$$T = \m{
a_{0} & a_{-1} & a_{-2} & \ldots & \ldots  &a_{-(n-1)}  \\
  a_{1} & a_0  & a_{-1} &  \ddots   &  &  \vdots \\
  a_{2}    & a_{1} & \ddots  & \ddots & \ddots& \vdots \\ 
 \vdots &  \ddots & \ddots &   \ddots  & a_{-1} & a_{-2}\\
 \vdots &         & \ddots & a_{1} & a_{0}&  a_{-1} \\
a_{n-1} &  \ldots & \ldots & a_{2} & a_{1} & a_{0}
}.$$
Then
$$TZ_n - Z_nT = \m{
0&a_{0} & a_{-1} & a_{-2} & \ldots & \ldots    \\
0&  a_{1} & a_0  & a_{-1} &  \ddots   &   \\
0&  a_{2}    & a_{1} & \ddots  & \ddots & \ddots \\ 
\vdots& \vdots &  \ddots & \ddots &   \ddots  & a_{-1} \\
\vdots& \vdots &         & \ddots & a_{1} & a_{0} \\
0&a_{n-1} &  \ldots & \ldots & a_{2} & a_{1} \\
}-\m{
  a_{1} & a_0  & a_{-1} &  \ddots   &  &  \vdots \\
  a_{2}    & a_{1} & \ddots  & \ddots & \ddots& \vdots \\ 
 \vdots &  \ddots & \ddots &   \ddots  & a_{-1} & a_{-2}\\
 \vdots &         & \ddots & a_{1} & a_{0}&  a_{-1} \\
a_{n-1} &  \ldots & \ldots & a_{2} & a_{1} & a_{0} \\
0 & \cdots & \cdots & 0 & 0 & 0 \\
}$$
$$=\m{
-a_1  & & & & \\
-a_2 & & & & \\
-a_3 & & & & \\
\vdots & & & & \\
-a_{n-1} & & & & \\
0 & a_{n-1} & a_{n-2} & \cdots & a_2 & a_1 \\
}.$$
Thus the rank is still $n$ in general.

b) Let $V$ be an $n\times n$ unit upper-triangular Toeplitz matrix. Show that $Z_n = VZ_n V^{-1}$.

\begin{proof}
It suffices to show $Z_nV = VZ_n$. Let
$$V = \m{
1& v_1 & v_2 & \cdots & v_{n-1} \\
 & 1 & v_1 & \cdots & v_{n-2} \\
 &  & 1 & \cdots & v_{n-3} \\
 &  &  & \ddots & \vdots \\
 &  &  &  & 1 \\
}.$$
It is clear that
$$Z_nV = \m{
&1& v_1 & v_2 & \cdots & v_{n-2} \\
& & 1 & v_1 & \cdots & v_{n-3} \\
& &  & 1 & \cdots & v_{n-4} \\
& &  &  & \ddots & \vdots \\
& &  &  &  & 1 \\
& &  &  &  & \\
} =VZ_n.$$
\end{proof}

c) Find all solutions $X_R$ of the equation $X_RZ_n - Z_nX_R = 0$. What is the dimension of the solution space?

{\it Solution.} For any solution $X_R$, it is clear that $(X_R)_{i,j-1} = (X_R)_{i+1,j}$ for any $0\leq i \leq n-1$ and $1\leq j\leq n$. That is, $X_R$ must be a Toeplitz matrix.
Based on a), for any solution $X_R$, we have 
$$a_1 = a_2 = a_3 = \cdots = a_{n-1} = 0.$$
That is, $X_R$ must be strictly upper triangular and Toeplitz. There are $n$ variables to define the solution $X_R$:
$$a_0, a_{-1}, a_{-2}, \cdots, a_{-(n-1)}.$$
Thus the dimension of the solution space is $n$.

d) Find all solutions $Y_L$ of the equation $Y_LZ_n^T - Z_n^TY_L = 0$. What is the dimension of the solution space?

{\it Solution.} It suffices to solve the equation $Z_nY_L^T - Y_L^T Z_n = 0$. By c), we know $Y_L^T$ must be strictly upper triangular Toeplitz. Hence the solution $Y_L$ must be strictly lower triangular Toeplitz. The dimension of the solution space is still $n$.

e) Consider the equation $XZ_n - Z_nX = Y$. Show that this equation has a solution $X$ iff $\mathrm{trace}(Y_L^TY)=0$, where $Y_L$ is from part d).

\begin{proof}
We first show that if the equation has a solution $X$, then the trace of $Y_L^TY$ is zero.
Clearly the most left-bottom entry of $Y$ must be 0: $Y_{n,1} = 0$, since the $(n,1)$ entry of $XZ_n - Z_nX$ must be 0 for any $X$. That is,
$$Y=\m{
-x_{2,1} & x_{1,1}-x_{2,2} & x_{1,2}-x_{2,3} & \cdots & x_{1,n-1}-x_{2,n} \\
-x_{3,1} & x_{2,1}-x_{3,2} & x_{2,2}-x_{3,3} & \cdots & x_{2,n-1}-x_{3,n} \\
-x_{4,1} & x_{3,1}-x_{4,2} & x_{3,2}-x_{4,3} & \cdots & x_{3,n-1}-x_{4,n} \\
\vdots & \vdots & \vdots & \ddots & \vdots \\
-x_{n,1} & x_{n-1,1}-x_{n,2} & x_{n-1,2}-x_{n,3} & \cdots & x_{n-1,n-1}-x_{n,n} \\
0 & x_{n,1} & x_{n,2}  & \cdots & x_{n,n-1} \\
}.$$
Note that the trace of $Y$ is zero. Since $Y_L^T$ is strictly upper triangular Toeplitz, the trace of $Y_L^TY$ is 
$$\mathrm{trace}(Y_L^TY) = (Y_L)_{1,1}\mathrm{trace}(Y) = 0.$$

We next show that if $\mathrm{trace}(Y_L^TY)=0$, then the equation must have a solution $X$. Since
$$0 = \mathrm{trace}(Y_L^TY) = \sum_{i,j} (Y_L)_{j,i}Y_{j,i} = \sum_{i\leq j}(Y_L)_{j,i}Y_{j,i},$$
and there are $n$ independent variables in $Y_L$, to make sure the sum is zero for arbitrary $Y_L$,
we have $\mathrm{trace}(Y)=0$ and $Y_{n,1}=0$. Thus we can write $Y$ as
$$Y=\m{
-x_{2,1} & x_{1,1}-x_{2,2} & x_{1,2}-x_{2,3} & \cdots & x_{1,n-1}-x_{2,n} \\
-x_{3,1} & x_{2,1}-x_{3,2} & x_{2,2}-x_{3,3} & \cdots & x_{2,n-1}-x_{3,n} \\
-x_{4,1} & x_{3,1}-x_{4,2} & x_{3,2}-x_{4,3} & \cdots & x_{3,n-1}-x_{4,n} \\
\vdots & \vdots & \vdots & \ddots & \vdots \\
-x_{n,1} & x_{n-1,1}-x_{n,2} & x_{n-1,2}-x_{n,3} & \cdots & x_{n-1,n-1}-x_{n,n} \\
0 & x_{n,1} & x_{n,2}  & \cdots & x_{n,n-1} \\
}.$$
Here the $x_{i,j}$ give the solution $X$. Note that $x_{1,n}$ has no requirement and thus can be any real number. Hence there are $n^2$ variables, but there are only $n^2-2$ independent linear equations. Thus the nullspace of $X$ has dimension of 2. That is, all the solutions are in the form $X+\lambda Y_L+\theta K$, where $\lambda,\theta\in\mathbb{R}$ and $K$ has only one non zero entry: $K_{1,n}=1$.
\end{proof}

f) Give an efficient algorithm to find a particular solution $X_p$ such that $X_pZ_n - Z_nX_p = Y$, assuming $Y$ is such that a solution exists.

{\it Solution.} The construction of $Y$ given in e) motivates Algorithm \ref{alg}.

\begin{algorithm}
\centering
\begin{minipage}{0.95\textwidth}
\begin{algorithmic}
\Require Matrix $Y_{n\times n}$.
\Ensure Matrix $(X_p)_{n\times n}$ s.t. $X_pZ_n - Z_nX_p = Y$.
\For {$2\leq i \leq n$}
	\State $x_{i,1} \gets -Y_{i-1,1}$,
\EndFor
\For {$2\leq i \leq n$}
	\State $x_{n,i} \gets Y_{n,i+1}$,
\EndFor
\For {$1\leq i \leq n-1$}
	\State $x_{1,i} \gets 1$,
\EndFor
\For {$2\leq i \leq n$}
	\For {$2\leq j\leq n$}
		\State $x_{i,j} \gets x_{i-1,j-1} - Y_{i-1,j}$,
	\EndFor
\EndFor
\State $x_{1,n} \gets 1$,
\State {\bf return} $X := (x_{i,j})_{n\times n}$.
\end{algorithmic}
\end{minipage}
\caption{The algorithm to compute one specific solution $X_p$ such that $X_pZ_n - Z_nX_p = Y$.}
\label{alg}
\end{algorithm}

It is easy to see the complexity of Algorithm \ref{alg} is $O(n^2)$.

g) Find all solutions $X$ of $XZ_n - Z_nX=Y$, assuming that $Y$ is such that a solution exists.

{\it Solution.} Algorithm \ref{alg} in f) gives one specific solution $X_p$ s.t. $(X_p)_{1,i}=1$ for $1\leq i\leq n$. The discussions in e) give the fact that all the solutions are in the form $X = X_p + \lambda Y_L + \theta K$, in which $\lambda,\theta\in\mathbb{R}$ and $K$ has only one non zero entry: $K_{1,n}=1$.

h) Given $Y$, show that the equation $X- Z_n^T XZ_n = Y$ always has a unique solution $X$ and give an efficient algorithm to compute $X$ from $Y$.

\begin{proof}
Let $X = (x_{i,j})_{n\times n}$. Hence based on $X- Z_n^T XZ_n = Y$, we have
$$Y=\m{
x_{1,1} & x_{1,2} & x_{1,3} & \cdots & x_{1,n} \\
x_{2,1} & x_{2,2}-x_{1,1} & x_{2,3}-x_{1,2} & \cdots & x_{2,n}-x_{1,n-1} \\
x_{3,1} & x_{3,2}-x_{2,1} & x_{3,3}-x_{2,2} & \cdots & x_{3,n}-x_{2,n-1} \\
\vdots & \vdots & \vdots & \ddots & \vdots \\
x_{n,1} & x_{n,2}-x_{n-1,1} & x_{n,3}-x_{n-1,2} & \cdots & x_{n,n}-x_{n-1,n-1} \\
}.$$
It is easy to see that the solution of $X$ can be uniquely determined given the matrix $Y$ above. The computing algorithm can be given analogously to Algorithm \ref{alg}. That is Algorithm \ref{alg2}.

\begin{algorithm}
\centering
\begin{minipage}{0.95\textwidth}
\begin{algorithmic}
\Require Matrix $Y_{n\times n}$.
\Ensure Matrix $X_{n\times n}$ s.t. $X- Z_n^T XZ_n = Y$.
\For {$1\leq i \leq n$}
	\State $x_{i,1} \gets Y_{i,1}$,
\EndFor
\For {$2\leq i \leq n$}
	\State $x_{1,i} \gets Y_{1,i}$,
\EndFor
\For {$2\leq i \leq n$}
	\For {$2\leq j\leq n$}
		\State $x_{i,j} \gets Y_{i,j} + x_{i-1,j-1}$,
	\EndFor
\EndFor
\State {\bf return} $X_p := (x_{i,j})_{n\times n}$.
\end{algorithmic}
\end{minipage}
\caption{The algorithm to compute one specific solution $X$ s.t. $X- Z_n^T XZ_n = Y$.}
\label{alg2}
\end{algorithm}

It is easy to see the complexity of Algorithm \ref{alg2} is also $O(n^2)$.

\end{proof}

\section{Problem 4}
Find the Jordan decomposition of the $n\times n$ lower-triangular matrix
$$
\m{
1 & 0 & 0 & \cdots & \cdots & 0 \\
1 & 1 & 0 & \cdots & \cdots & 0 \\
1 & 0 & 1 & \ddots &  & \vdots \\
\vdots & \vdots & \ddots & \ddots & \ddots & \vdots \\
1 & 0 & 0 & \cdots & 1 & 0 \\
1 & 1 & \cdots & 1 & 1 & 1 \\
}.
$$
Note that the first column, last row and main diagonal are filled with 1's. The rest of the matrix is filled with 0's.

\begin{proof}
Let the decomposition be $M = VJV^{-1}$. We first check $J$. Let $R = M - I$. Clearly the rank of $R$ is 2, implying the dimension of the nullspace of $R$ is $n-2$. That is, there are $n-2$ Jordan blocks in $J$. Hence
$$J = \m{
1&1& & & & & \\
 &1&1& & & & \\
 & &1& & & & \\
 & & &1& & & \\
 & & & &1 & & \\
 & & & & &\ddots & \\
 & & & & & &1 \\
}.$$
The computation in the last question of HW 7, Track 2 gives the case when $n=4$. Then
$$V = \m{
0&0&1&0&0&0\\
0&1&0&-1/4&-1/4&-1/4\\
0&1&0&-1/4&-1/4&3/4\\
0&1&0&3/4&-1/4&-1/4\\
0&1&0&-1/4&3/4&-1/4\\
4&1&0&-1/4&-1/4&-1/4\\
},\quad
V^{-1} = \m{
0&-1/4&0&0&0&1/4\\
0&1/4&1/4&1/4&1/4&0\\
1&0&0&0&0&0\\
0&-1&0&1&0&0\\
0&-1&0&0&1&0\\
0&-1&1&0&0&0\\
}.$$
The cases $n=2, 3$ can be computed similarly. Thus denoting $m=n-2$, we conjecture that for any $n$, it should be
$$V=\m{
0 & 0 & 1 & 0 & 0 & \cdots & 0 & 0 \\
0 & 1 & 0 & -1/m & -1/m & \cdots & -1/m & -1/m \\
0 & 1 & 0 & -1/m & -1/m & \cdots & -1/m & 1-1/m \\
0 & 1 & 0 & 1-1/m & -1/m & \cdots & -1/m & -1/m \\
0 & 1 & 0 & -1/m & 1-1/m & \cdots & -1/m & -1/m \\
\vdots & \vdots & \vdots & \vdots & \vdots & \ddots & \vdots \\
0 & 1 & 0 & -1/m & -1/m & \cdots & 1-1/m & -1/m \\
n-2 & 1 & 0 & -1/m & -1/m & \cdots & -1/m & -1/m \\
},$$
and
$$V^{-1}=\m{
0 & -1/m & 0 & 0 & 0 & \cdots & 0 & 1/m \\
0 & 1/m & 1/m & 1/m & 1/m & \cdots & 1/m & 0 \\
1 & 0 & 0 & 0 & 0 & \cdots & 0 & 0 \\
0 & -1 & 0 & 1 & 0 & \cdots & 0 & 0 \\
0 & -1 & 0 & 0 & 1 & \cdots & 0 & 0 \\
\vdots & \vdots & \vdots & \vdots & \vdots & \ddots & \vdots \\
0 & -1 & 0 & 0 & 0 & \cdots & 1 & 0 \\
0 & -1 & 1 & 0 & 0 & \cdots & 0 & 0 \\
}.$$
A straightforward computation verifies the fact $VV_{-1} = I_n$. Also we have
$$JV^{-1} = \m{
1&1& & & & & \\
 &1&1& & & & \\
 & &1& & & & \\
 & & &1& & & \\
 & & & &1 & & \\
 & & & & &\ddots & \\
 & & & & & &1 \\
}\m{
0 & -1/m & 0 & 0 & 0 & \cdots & 0 & 1/m \\
0 & 1/m & 1/m & 1/m & 1/m & \cdots & 1/m & 0 \\
1 & 0 & 0 & 0 & 0 & \cdots & 0 & 0 \\
0 & -1 & 0 & 1 & 0 & \cdots & 0 & 0 \\
0 & -1 & 0 & 0 & 1 & \cdots & 0 & 0 \\
\vdots & \vdots & \vdots & \vdots & \vdots & \ddots & \vdots \\
0 & -1 & 0 & 0 & 0 & \cdots & 1 & 0 \\
0 & -1 & 1 & 0 & 0 & \cdots & 0 & 0 \\
}$$
$$
=\m{
0 & 0 & 1/m & 1/m & 1/m & \cdots & 1/m & 1/m \\
0 & 1/m & 1/m & 1/m & 1/m & \cdots & 1/m & 0 \\
1 & 0 & 0 & 0 & 0 & \cdots & 0 & 0 \\
0 & -1 & 0 & 1 & 0 & \cdots & 0 & 0 \\
0 & -1 & 0 & 0 & 1 & \cdots & 0 & 0 \\
\vdots & \vdots & \vdots & \vdots & \vdots & \ddots & \vdots & \vdots \\
0 & -1 & 0 & 0 & 0 & \cdots & 1 & 0 \\
0 & -1 & 1 & 0 & 0 & \cdots & 0 & 0 \\
}.
$$
Then it is straightforward to verify that 
$$VJV^{-1} = M=\m{
1 & 0 & 0 & \cdots & \cdots & 0 \\
1 & 1 & 0 & \cdots & \cdots & 0 \\
1 & 0 & 1 & \ddots &  & \vdots \\
\vdots & \vdots & \ddots & \ddots & \ddots & \vdots \\
1 & 0 & 0 & \cdots & 1 & 0 \\
1 & 1 & \cdots & 1 & 1 & 1 \\
}.$$
\end{proof}

\end{document}
